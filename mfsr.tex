
\documentclass[12pt, a4paper]{amsart}
\usepackage{amssymb}
\usepackage{amsthm}
\usepackage{amsfonts}
\usepackage{url}
\usepackage{tikz}
\usetikzlibrary{calc}
\newtheorem{lemma}{\bf Lemma}[section]
\newtheorem{proposition}[lemma]{\bf Proposition}
\newtheorem{definition}[lemma]{\bf Definition}
\parindent = 0mm
\parskip = 2mm
\begin{document}
\title{Majority Feedback Shift Registers}
\author{Dominic van der Zypen}
%\address{Swiss Armed Forces, CH-3003 Bern, Switzerland}
\email{dominic.zypen@gmail.com}
\begin{abstract}
This note we introduce ``majority feedback shift registers''. The
new bit is calculated via a majority voting from the binary values
	of the taps (fixed bit positions): If the majority of the 
	taps are set to $0$, the output bit is set to $1$, and vice
	versa. This is a highly non-linear function that is appealing
	because of its simplicity.
\end{abstract}
%----------------------
\maketitle
%----------------------
\section{Introduction}

First, we define general feedback shift registers
in a mathematically proper way.

\begin{definition}
\emph{
Let $n\in\mathbb{N}$ be a positive integer.
A function $$r:\{0,1\}^n \to \{0,1\}^n$$ is said to be
an {\em $n$-bit feedback shift register} 
if there is a one-bit-output map $$f : \{0,1\}^n \to \{0,1\}$$ such that 
$$r(b_0, b_1,\ldots, b_{n-1})
=  \big(b_1, b_2, \ldots, f(b_0, b_1,\ldots, b_{n-1})\big) \text{ for all } 
	b\in\{0,1\}.$$ 
The map $f$ is said to be the {\em feedback function}.
}
\end{definition}
Recall the basic mechanism of linear feedback shift registers (LFSR): 
Given a fixed set of taps (bit positions), the ``new'' bit is calculated 
by applying $\oplus$ ({\tt XOR})\footnote{in {\tt C} and many other 
languages, the caret {\string^} is used for bitwise {\tt XOR}}
to those taps. Then the new bit 
is inserted at the left end of the binary string to be manipulated,
and the other bits are shifted. 
Linear feedback shift registers (LFSR's) have a long history
and plenty of applications, for instance as pseudo-number generators,
see \cite{golo}.

The fact that XOR is a linear 
function\footnote{when seen as an operation in vector spaces over
the field $\mathbb{Z}/2\mathbb{Z}$} 
is seen as a disadvantage for some applications. For instance,
LFSR's are not cryptographically secure as they yield to
analysis using a set of linear equations, see \cite{mar}.

There have been several proposals to introduce non-linearity
to the feedback function $f:\{0,1\}^n \to \{0,1\}$.
The resulting feedback shift registers are often slow, though.

There is one known 
way\footnote{ \url{https://en.wikipedia.org/wiki/Nonlinear-feedback_shift_register}}
of constructing nonlinear feedback shift registers having period
$2^n = |\{0,1\}^n|$, thus cycling through all possible bit-strings
of length $n$. But the construction of other large NLFSRs with 
guaranteed long periods remains an intriguing open problem.

%----------------------------
\section{Let the majority decide}
As a new proposal, instead of XORing a few taps as it is done
in LFSR's,
fix an {\em odd number of taps} and  
{\bf hold a vote} for the corresponding binary values: 
If the majority of the taps is set to $0$, the new bit is $1$,
and vice versa. This ``inversion'' (a majority of $0$'s sets
the output to $1$ and vice versa) serves the purpose to 
avoid constantness in the generated bit-string. So we should
call the such a function an {\em inverse majority vote}.

But the outcome is nevertheless decided by a majority.

For the sake of completeness, we rigorously define
the concept of an (inverse) majority feedback function for 
an odd number of taps.

\begin{definition}
\emph{
Let $n>1$, and let $T\subseteq 
\{0,\ldots,n-1\}$ with $|T|$ odd be the 
``the set of taps''. Let $\text{maj}_T:\{0,1\}^n\to \{0,1\}$ be
defined by 
    \begin{quote}$\text{maj}_T(b) = 1$ if $\; |T| <
    2\cdot\big|\{i\in T: b_i = 0\}\big|$, \newline
    and $\text{maj}_T(b) = 0$ otherwise.
\end{quote}
Then $\text{maj}_T: \{0,1\}^n\to\{0,1\}$ is said to be
the {\em (inverse) majority number function} on $T$.
The corresponding feedback shift register is said to be
the {\em majority feedback shift register (MFSR)}, 
denoted by $r_T$ and it is defined by 
	$$(b_0,\ldots,b_{n-1})\mapsto 
	\big(b_1, b_2,\ldots, \text{maj}_T(b_0,
	\ldots,b_{n-1})\big).$$
}
\end{definition}
%---------------------------
\section{Open questions regarding MFSR}
As the concept of majority voting in shift register seems 
to be quite new, a few open questions arise. 
\begin{enumerate}
\item For which positive integers $n\in\mathbb{N}$ can the 
tap set $T$ be chosen such that the period
of the corresponding MFSR $r_T$
	has period $2^n$?
\item Is there for all positive integers $n$ an 
	MFSR with period $O(2^n)$?
\item (Soft question) Can the tap set $T$ be chosen such that
	$r_T:\{0,1\}^n\to\{0,1\}^n$ is cryptographically secure
		for large enough $n$?
\end{enumerate}
%---------------------------
\section{{\tt C} code for a short and fast implementation}
In the following we give the {\tt C} code for MFSR for {\tt uint16\_t} 
numbers. It can easily be generalised to other numbers of bits.

The code\footnote{\url{https://github.com/dominiczypen/Majority_Feedback_Shift_Register/blob/main/mfsr.c}} relies heavily on population count (counting numbers of $1$'s
in a bit string). I have written this in {\tt C}, but it would be
better to use the corresponding {\tt x86} primitive. 

\begin{footnotesize}
\begin{verbatim}
/* Author: Dominic van der Zypen
 * File: mfsr.c
 */

#include <stdio.h>
#include <stdlib.h>
#include <stdint.h>   //  --> for using uint16_t

char popcount(uint16_t x);
char majority_vote(uint64_t x, char* taps, char num_of_taps);
void mfsr(uint16_t* x, char* taps, char num_of_taps);
//--------------------
char popcount(uint16_t x)
  // counts number of 1's in binary representation of x
{
  char count = 0;
  while (x)
  {
    if (x & 1) {count++;}
    x = x >> 1;
  }
  return count;
}
//--------------------
char majority_vote(uint64_t x, char* taps, char num_of_taps)
  // taps need to be distinct bit positions \in \{0, ... , 15\}
  // num_of_taps needs to be odd, so we get no ties.
  // return 1 if the 0's have majority, 0 otherwise
{
  uint16_t mask = 0;

  // build mask: 1 at the taps positions, 0 otherwise

  char i = 0;
  while (i < num_of_taps)
  {
    mask = mask | (1 << *(taps+i));
    i++;
  }
  char majority1 = ((popcount(mask & x) << 1) > num_of_taps);
    // majority1 == 1 if and only if there are more 1 than 0's in masked x
  return 1 - majority1;
}
//--------------------
void mfsr(uint16_t* up, char* taps, char num_of_taps)
  // new bit is determined with majority rule and then inserted after shift
{
  char my_new_bit = majority_vote(*up, taps, num_of_taps);
  *up = (my_new_bit << 15) | ((*up) >> 1); 
          // ... << 15 -> put new bit at left end!
}
//--------------------
void print_uint16_bin(uint16_t x)
{
  char i = 0;
  char my_bit = 0;
  while (i < 16)
  {
    my_bit = (x >> 15) & 1;
    printf("%d", my_bit);
    x = x << 1;
    i++;
    if (!(i & 3)) {printf(" ");} // formatting; "& 3" is "modulo 4 != 0
  }
  printf("\n");
}
//--------------------
int main()
{
  char* taps = (char*)malloc(sizeof(char) * 3);
  *taps = 0; *(taps + 1) = 4; *(taps + 2) = 7; // example taps (3 in total)
  char i = 0;
  uint16_t x = 0x95c1; // 4 nibbles -> 16 bit
  while (i < 20)
  {
    print_uint16_bin(x);
    mfsr(&x, taps, 3);
    i++;
  }
  return 0;
}
\end{verbatim}
\end{footnotesize}
\newpage
\begin{thebibliography}{1}
	\bibitem{golo} Golomb, S.W. (1967). {\em Shift register sequences.} 
		Laguna Hills, Calif.: Aegean Park Press. 
	\bibitem{mar} Martinez L.H., Khursheed S., Reddy S.M. (2019).
		{\em LFSR generation for high test coverage and 
		low hardware overhead.} 
		IET Computers \& Digital Techniques.
\end{thebibliography}
\end{document}
